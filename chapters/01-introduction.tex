
\chapter{Introduction}
Cache memory is a crucial element in improving and maintaining the performance of modern computer systems, applicable in contexts ranging from relatively small embedded cores
and DSP processors to advanced, highly-performant processors. Its significance spans across various applications, from enhancing the efficiency of everyday computing tasks to
supporting the complex demands of scientific computations. The effective utilization and optimization of cache memory can lead to significant improvements in processing speed and
overall system performance.

\section{Motivation and goals of the thesis}
Cache memory plays a critical in modern computer systems. As systems become more complex and performance demands increase, optimizing cache performance becomes
essential for both high-performance computing applications and smaller, time-sensitive workloads. Despite significant advancements in cache design, there remains a continual need
to explore new methodologies and tools to profile and enhance cache usage effectively.

\vspace{10px}
\noindent The primary goals of this thesis include:

\begin{itemize}
  \item Evaluation of the effectiveness of trace-based approaches for profiling CPU cache usage and performance: Trace-based methods will be examined for their ability to provide
    insights into cache behavior.
  \item Providing an in-depth understanding of the core concepts of CPU caches: This work aims to offer a comprehensive exploration of CPU cache mechanisms, including their role in
    the memory architecture, placement policies, replacement strategies.
  \item Assessment of algorithm optimizations and various hardware configurations on cache usage: This work will analyze the impact of both various cache configurations
    and algorithm optimizations in the context of cache usage, with the goal of identifying the most effective strategies for improving its utilization.
  \item Proposal of improvements and future work based on the findings: The results from this work will be used to suggest further research and developments related to the concepts
explored in this thesis.

\end{itemize}

\section{Thesis organization}
The work is divided into the following chapters:
\begin{itemize}
  \item Chapter 2 covers the \textbf{basics of cache}, including a detailed description of modern \textbf{computer architecture} and design.
  \item Chapter 3 provides an in-depth review of the \textbf{state of the art} in cache memory simulators.
  \item Chapter 4 describes the implementation of the \textbf{cache model}, and its integration with the \textbf{Renode} simulation framework
  \item Chapter 5 describes the benchmarks and the test binaries, and presents their results
  \item Chapter 6 concludes the thesis and suggests \textbf{future work}
\end{itemize}
