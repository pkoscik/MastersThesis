
\chapter{Background}


\section{System emulation}
%

\section{Caches and memory hierarchies}
\subsection{Overview of memory hierarchy}
\subsection{Role of caches in system performance}
%

\section{CPU caches}
\subsection{Basic configuration parameters}
%
\subsection{Placement policies}
Cache placement policies determine where a specific memory block can be loaded into
the cache. The choice of placement policy influences the cache architecture and
its control logic - affecting the overall complexity and performance of the system.
Each policy involves trade-offs between speed, by the means of reducing cache misses and
thrashing, and hardware costs related to the size and design of the hardware.

% TODO: explain how cache is adressed, how entry is split etc

\subsubsection{Fully associative cache}
In the fully associative cache each \textit{cache line} can hold a copy of
\textit{any memory location}.
% TODO: image, diagram

\noindent This configuration minimizes the chances of cache misses due to conflicts,
potentially improving performance. However, this type of cache requires
complex hardware for searching and managing, as it needs to check all entries
simultaneously. This results in higher power consumption and increased die size.

\subsubsection{Set associative cache}
The set associative cache introduces a concept of a \textit{set} - a collection
of more than one cache line.
% TODO: image, diagram

\subsubsection{Directly mapped cache}
In the directly mapped cache each \textit{cache line} can hold a copy of
a single \textit{tag}. This policy is equivalent to the set associative cache, with
1-way associativity.
% TODO: image, diagram

%
\subsection{Replacement policies}
\subsubsection{Queue based}
\subsubsection{Recency based}
\subsubsection{Frequency based}
%
\subsection{Cache coherency}
\subsubsection{Direct memory access}
\subsubsection{Symmetric multiprocessing}
%
