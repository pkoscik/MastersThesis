
\chapter{Background}


\section{System emulation}

% TODO: add footnote explaining that emulation and simulation will be used interchangebly
% TODO: please for the love of GOD refactor/proofread this section
System emulation is a technique of modeling and imitating the hardware of one
system on another system. The range of emulation depends on the use case; some
emulators, such as Wine or QEMU in KVM mode, only emulate system-level calls,
executing the rest of the application natively. Another example of emulation
is software like VirtualBox, VMWare, or Microsoft Hyper-V, which uses host
processor virtualization extensions to run the guest operating system in a
supervised environment, allowing for more separation between the guest software.
More advanced emulators, such as the Renode framework, can emulate the entire
platform, including processors and peripherals of another architecture. Another
notable example of such an emulator is QEMU's system mode emulation and Intel
Simics.

Emulation software has become a crucial part of developing modern-day software
and hardware. It finds uses in a variety of fields of software world, starting
from enabling cloud services providers with secure, reproducible, and isolated
execution environments. This task is mainly handled by the level one and
two hypervisors such as VMWare or Hyper-V. Another use for emulators is the
usage in continuous integration and delivery systems, where Kernel Virtual
Machines (KVM), such as QEMU KVM, are widely used to aid in deploying so-called
"runners", that execute testing and deployment software in a manageable and
scalable manner. Yet another scenario where emulation greatly enhances workflow
is the development of embedded/edge devices. Emulation software allows for
the development of software for non-PC platforms - such as microcontrollers,
SBCs or FPGA devices - without the need for physical hardware. This decoupling
has become important as development teams have grown larger and larger. This
streamlines the embedded software development process by further aligning it
with classical software development practices. Emulation solutions provide
first-class features such as reproducible CI pipelines, code coverage results
and other tools that have been taken for granted in the desktop development.

%

\section{Caches and memory hierarchies}

% TODO: some bullshit about how computers work here

\subsection{Overview of memory hierarchy}
\subsection{Role of caches in system performance}
%

\section{CPU caches}
\subsection{Basic configuration parameters}
%
\subsection{Placement policies}
Cache placement policies determine where a specific memory block can be loaded into
the cache. The choice of placement policy influences the cache architecture and
its control logic - affecting the overall complexity and performance of the system.
Each policy involves trade-offs between speed, by the means of reducing cache misses and
thrashing, and hardware costs related to the size and design of the hardware.

% TODO: explain how cache is adressed, how entry is split etc

\subsubsection{Fully associative cache}
In the fully associative cache each \textit{cache line} can hold a copy of
\textit{any memory location}.
% TODO: image, diagram

\noindent This configuration minimizes the chances of cache misses due to conflicts,
potentially improving performance. However, this type of cache requires
complex hardware for searching and managing, as it needs to check all entries
simultaneously. This results in higher power consumption and increased die size.

\subsubsection{Set associative cache}
The set associative cache introduces a concept of a \textit{set} - a collection
of more than one cache line.
% TODO: image, diagram

\subsubsection{Directly mapped cache}
In the directly mapped cache each \textit{cache line} can hold a copy of
a single \textit{tag}. This policy is equivalent to the set associative cache, with
1-way associativity.
% TODO: image, diagram

%
\subsection{Replacement policies}
\subsubsection{Queue based}
\subsubsection{Recency based}
\subsubsection{Frequency based}
%
\subsection{Cache coherency}

The goal of cache coherency is to maintain a consistent state between two or
more separate cache memories. This process can take place both in single-core
systems—in which the state may be synchronized between different levels of cache
- and in multicore processors, where the caches are kept in sync between multiple
compute nodes.

\subsubsection{Direct memory access}

Direct memory access, also referred to as \textit{DMA}, is a mechanism that
allows peripheral devices, such as expansion cards used in personal computers
and various peripherals on embedded devices, to directly access main system memory,
bypassing the CPU.

\subsubsection{Symmetric multiprocessing}
%
