
\chapter{Background}


\section{System emulation}
%

\section{Caches and memory hierarchies}
\subsection{Overview of memory hierarchy}
\subsection{Role of caches in system performance}
%

\section{CPU caches}
\subsection{Basic configuration parameters}
%
\subsection{Placement policies}
Cache placement policies determine where a specific memory block can be loaded into
the cache. The choice of placement policy influences the cache architecture and
its control logic - affecting the overall complexity and performance of the system.
Each policy involves trade-offs between speed, by the means of reducing cache misses and
thrashing, and hardware costs related to the size and design of the hardware.

\subsubsection{Fully associative cache}
In the fully associative cache, each \textit{cache line} can hold a copy of
\textit{any memory location}.
% TODO: image, diagram



\subsubsection{Set associative cache}
The set associative cache introduces a concept of a \textit{set} - a collection
of more than one cache line.
% TODO: image, diagram

\subsubsection{Directly mapped cache}
In the directly mapped cache, each \textit{cache line} can hold a copy of
a single \textit{tag}.
% TODO: image, diagram

%
\subsection{Replacement policies}
\subsubsection{Queue based}
\subsubsection{Recency based}
\subsubsection{Frequency based}
%
\subsection{Cache coherency}
\subsubsection{Direct memory access}
\subsubsection{Symmetric multiprocessing}
%
