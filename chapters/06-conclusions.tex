
\chapter{Conclusions}

This thesis has explored the efficacy of using trace-based approaches for evaluating CPU cache usage and performance. Several key points should be noted:

\subsubsection*{Effectiveness of trace-based evaluation}

Trace-based evaluation has been proven to be an effective solution for profiling CPU cache performance.
By utilizing trace data, it is possible to gain detailed insights into cache behavior, such as cache hits, misses, and the overall hit-to-miss ratio (HMR). This method allows
for precise analysis of how different cache configurations impact system performance, enabling the identification of bottlenecks and opportunities for optimization.
For example, the data from the matrix multiplication benchmarks clearly showed how different block sizes and cache configurations affect the number of cache misses and the
efficiency of cache usage.

\subsubsection*{Impact of cache size and configuration}

The Linux kernel payload benchmark results observed and described in the section (\ref{sec:linux_boot_perf_conc}) confirm that cache size and configuration significantly impact CPU
performance. Factors such as cache size, block size, and associativity play crucial roles in determining system efficiency. Increasing the cache size generally leads to a reduction
in cache misses, which improves overall system efficiency, at the cost of more complicated hardware design. Additionally, optimizing block size and cache associativity can
further enhance performance.


\subsubsection*{Simulation and real hardware correlation}

This work has confirmed the correlation of the CPU cache behavior between the real hardware platform and the simulated virtual platform and cache model implemented in this work. By
using virtual platforms, developers are provided with an immediate feedback on the impact of code changes on cache performance. This proactive approach to performance optimization
reduces the need for extensive post-deployment tuning, leading to faster and easier development processes.

\subsubsection*{Practical implications for system design}

The thesis provides a set of findings that have practical implications for system design:

\begin{itemize}
	\item \textbf{Improved cache utilization:} the insights gained from trace-based evaluations can be used to improve cache utilization in various systems. By having a toolset to
		quantitatively and qualitatively measure cache performance, developers can optimize and improve their code and algorithms.
	\item \textbf{Informed hardware design:} the methodology implemented in this work can be adopted for gathering better understanding of the various cache hardware design elements -
		improving the design process
	\item \textbf{Benchmarking and validation:} the established benchmarks and methodologies can be adopted for future hardware validation and implementation of 
\end{itemize}

\section{Future work}

Future research could extend this work by exploring trace-based evaluation in more complex and diverse computing environments. Areas for further investigation include:

\subsubsection*{Multi-core and multi-level cache environments}

This work has not examined the cache behavior in multi-core systems. Further work could be done to extend the cache model with cache coherency protocols, such as MESI or MOESI.
Additionally, the cache model implemented in this work could be further enhanced by adding support for multi-level caches (L2, L3).


\subsubsection*{Advanced cache models}

The model implemented in this work only implements the traditional cache behavior. Further work can be put into implementing advanced cache mechanisms, such as
the Trace Assisted Caching \cite{usingtrace} and dynamic line replacement algorithms \cite{dynamiceviction}.

\subsubsection*{Hardware integration}
