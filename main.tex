%%%%%%%%%%%%%%%%%%%%%%%%%%%%%%%%%%%%%%%%%%%%%%%%%%
%% Bachelor's & Master's Thesis Template        %%
%% Copyleft by Dawid Weiss & Marta Szachniuk    %%
%% Faculty of Computing and Telecommunication   %%
%% Poznan University of Technology, 2020        %%
%%%%%%%%%%%%%%%%%%%%%%%%%%%%%%%%%%%%%%%%%%%%%%%%%%


% Szkielet dla pracy licencjackiej pisanej w języku polskim.

\documentclass[english,masters,a4paper,oneside]{ppfcmthesis}


\usepackage[utf8]{inputenc}
\usepackage[OT4]{fontenc}

% Fix footnotes to the bottom of the page
\usepackage[bottom, perpage]{footmisc}

% Better table formatting
\usepackage{makecell}               % Makecell for breaking a long text
\renewcommand{\cellalign}{tl}       % Makecell left alignment
\renewcommand{\arraystretch}{1.75}  % More vertical padding

% Code snippets
\usepackage{listings}
\usepackage{color}

\usepackage{multicol}

\usepackage{caption}
\usepackage{subcaption}
\usepackage{multirow}

\usepackage{pdfpages}

\definecolor{codegreen}{rgb}{0.00,0.60,0.00}
\definecolor{preprocesorbrown}{rgb}{0.50,0.25,0.23} 
\definecolor{codelightblue}{rgb}{0.30,0.70,0.60}

\lstset{frame=tb,
  aboveskip=3mm,
  belowskip=3mm,
  showstringspaces=false,
  columns=flexible,
  basicstyle={\small\ttfamily},
  numbers=none,
  numberstyle=\tiny\color{gray},
  keywordstyle=\color{codelightblue},
  commentstyle=\color{codegreen},
  stringstyle=\color{preprocesorbrown},
  breaklines=true,
  breakatwhitespace=true,
  tabsize=2,
  captionpos=b
}

% Lists without bulletpoints
\usepackage{enumitem}

\lstdefinestyle{lstC}
{
	language = C,
	keywordstyle     =   \color{blue}\ttfamily,
  	commentstyle     =   \color{codegreen}\ttfamily,
  	stringstyle      =   \color{red},
  	emphstyle        =   \color{codelightblue}\ttfamily,
  	directivestyle   =   \color{preprocesorbrown}\ttfamily,
     morekeywords   =   {inline, target_ulong, uint32_t, uint64_t, no_inline, BOOL, _Bool},
}

\lstdefinestyle{lstCsharp} 
{
	language = C++, % No C# support
	keywordstyle     =\color{blue}\ttfamily,
  	commentstyle     =\color{codegreen}\ttfamily,
  	stringstyle      =\color{red},
  	emphstyle        =\color{codelightblue}\ttfamily,
  	directivestyle   =\color{preprocesorbrown}\ttfamily,
  	morekeywords     ={var,async,await,using,byte,string,get,set,uint, checked},
  	emph             ={Descendants,First,FormUrlEncodedContent,HttpClient,
  					   HttpWebRequest,HttpWebResponse,JObject,JProperty,
  					   KeyValuePair,Length,List,OfType,
  					   PropertyChangedEventHandler,Stream,StreamReader,
  					   Value,WebRequest,Where}
}

%--------------------------------------
% Strona tytułowa
%--------------------------------------

% Autorzy pracy, jeśli jest ich więcej niż jeden
% wstaw między nimi separator \and
\author
{%
   Patryk Kościk \album{144635}
}
\authortitle{}                                % Do not change.

\title
{%
   Analysis of Trace-Based Evaluation of Cache Usage on the Example of the Renode Framework
}

% Your supervisor comes here.
\ppsupervisor{dr inż. Mariusz Naumowicz} 

% Year of final submission (not graduation!)
\ppyear{2024}                                 


\begin{document}

% Front matter starts here
\frontmatter\pagestyle{empty}%
\maketitle\cleardoublepage%

%--------------------------------------
% Miejsce na kartę pracy dyplomowej
%--------------------------------------

\thispagestyle{empty}\vspace*{\fill}%
\begin{center}Tutaj będzie karta pracy dyplomowej;\\oryginał wstawiamy do wersji dla archiwum PP, w pozostałych kopiach wstawiamy ksero.\end{center}%
\vfill\cleardoublepage%


% \thispagestyle{empty}\vspace*{\fill}%

\begin{vplace}

\begin{center}
   \huge{\textit{Abstract}}
\end{center}

% TODO: proofread
In the context of modern computing, CPU cache plays a pivotal role in defining
system performance across both complex computing systems and edge/embedded solutions.
%
Consequently, considerable effort is being invested in enhancing cache
implementations and various optimizations, aimed at supporting both
extensive workloads, such as large machine learning models, and smaller, more
time-sensitive workloads, such as improving latencies in real-time operating
systems.
%
This work will focus on determining whether using trace-based approaches are an
effective solution for profiling CPU cache usage and performance.
%
Additionally, this study will examine the effects of cache size and configuration
on processing bottlenecks, offering insights into how these factors influence
overall system performance.

\end{vplace}

\newpage

\begin{vplace}

\begin{center}
   \huge{\textit{Streszczenie}}
\end{center}

% TODO: proofread
W kontekście nowoczesnych systemów komputerowych, pamięć podręczna procesora
odgrywa kluczową rolę w wydajności systemu, zarówno w złożonych
systemach obliczeniowych, jak i rozwiązaniach brzegowych/wbudowanych.
%
W związku z tym pokładane jest dużo pracy w ulepszanie implementacji pamięci
podręcznej i różne optymalizacje, mające na celu wspieranie zarówno złożonych
obliczeniowo zadań, takich jak duże modele uczenia maszynowego, jak i
mniejszych, bardziej wrażliwych na czas obciążeń, takich jak optymalizacja
opóźnień w systemach operacyjnych czasu rzeczywistego.
%
Ta praca skupi się na określeniu, czy wykorzystanie podejścia opartego
na śledzeniu jest skutecznym rozwiązaniem do profilowania wykorzystania i
wydajności pamięci podręcznej procesora.
%
Dodatkowo, to badanie zbada wpływ rozmiaru i konfiguracji pamięci podręcznej na
wąskie gardła przetwarzania, oferując wgląd w to, jak czynniki te wpływają na
ogólną wydajność systemu.

\end{vplace}

\newpage

%--------------------------------------
% Spis treści
%--------------------------------------

\pagenumbering{Roman}\pagestyle{ppfcmthesis}%
\tableofcontents* 
\cleardoublepage % Zaczynamy od nieparzystej strony

%--------------------------------------
% Rozdziały
%--------------------------------------

%Najwygodniej jeśli każdy rozdział znajduje się w oddzielnym pliku
\mainmatter%

\chapter{Introduction}
Cache memory is a crucial element in improving and maintaining the performance of modern computer systems, applicable in contexts ranging from relatively small embedded cores
and DSP processors to advanced, highly-performant processors. Its significance spans across various applications, from enhancing the efficiency of everyday computing tasks to
supporting the complex demands of scientific computations. The effective utilization and optimization of cache memory can lead to significant improvements in processing speed and
overall system performance.

\section{Motivation and goals of the thesis}
Cache memory plays a critical in modern computer systems. As systems become more complex and performance demands increase, optimizing cache performance becomes
essential for both high-performance computing applications and smaller, time-sensitive workloads. Despite significant advancements in cache design, there remains a continual need
to explore new methodologies and tools to profile and enhance cache usage effectively.

\vspace{10px}
\noindent The primary goals of this thesis include:

\begin{itemize}
  \item Evaluation of the effectiveness of trace-based approaches for profiling CPU cache usage and performance: Trace-based methods will be examined for their ability to provide
    insights into cache behavior.
  \item Providing an in-depth understanding of the core concepts of CPU caches: This work aims to offer a comprehensive exploration of CPU cache mechanisms, including their role in
    the memory architecture, placement policies, replacement strategies.
  \item Assessment of algorithm optimizations and various hardware configurations on cache usage: This work will analyze the impact of both various cache configurations
    and algorithm optimizations in the context of cache usage, with the goal of identifying the most effective strategies for improving its utilization.
  \item Proposal of improvements and future work based on the findings: The results from this work will be used to suggest further research and developments related to the concepts
explored in this thesis.

\end{itemize}

\section{Thesis organization}
The work is divided into the following chapters:
\begin{itemize}
  \item Chapter 2 covers the \textbf{basics of cache}, including a detailed description of modern \textbf{computer architecture} and design.
  \item Chapter 3 provides an in-depth review of the \textbf{state of the art} in cache memory simulators.
  \item Chapter 4 describes the implementation of the \textbf{cache model}, and its integration with the \textbf{Renode} simulation framework
  \item Chapter 5 describes the benchmarks and the test binaries, and presents their results
  \item Chapter 6 concludes the thesis and suggests \textbf{future work}
\end{itemize}


\chapter{Background}


\section{System emulation}

% TODO: add footnote explaining that emulation and simulation will be used interchangebly
% TODO: please for the love of GOD refactor/proofread this section
System emulation is a technique of modeling and imitating the hardware of one
system on another system. The range of emulation depends on the use case; some
emulators, such as Wine or QEMU in KVM mode, only emulate system-level calls,
executing the rest of the application natively. Another example of emulation
is software like VirtualBox, VMWare, or Microsoft Hyper-V, which uses host
processor virtualization extensions to run the guest operating system in a
supervised environment, allowing for more separation between the guest software.
More advanced emulators, such as the Renode framework, can emulate the entire
platform, including processors and peripherals of another architecture. Another
notable example of such an emulator is QEMU's system mode emulation and Intel
Simics.

Emulation software has become a crucial part of developing modern-day software
and hardware. It finds uses in a variety of fields of software world, starting
from enabling cloud services providers with secure, reproducible, and isolated
execution environments. This task is mainly handled by the level one and
two hypervisors such as VMWare or Hyper-V. Another use for emulators is the
usage in continuous integration and delivery systems, where Kernel Virtual
Machines (KVM), such as QEMU KVM, are widely used to aid in deploying so-called
"runners", that execute testing and deployment software in a manageable and
scalable manner. Yet another scenario where emulation greatly enhances workflow
is the development of embedded/edge devices. Emulation software allows for
the development of software for non-PC platforms - such as microcontrollers,
SBCs or FPGA devices - without the need for physical hardware. This decoupling
has become important as development teams have grown larger and larger. This
streamlines the embedded software development process by further aligning it
with classical software development practices. Emulation solutions provide
first-class features such as reproducible CI pipelines, code coverage results
and other tools that have been taken for granted in the desktop development.

%

\section{Caches and memory hierarchies}

% TODO: some bullshit about how computers work here

\subsection{Overview of memory hierarchy}
\subsection{Role of caches in system performance}
%

\section{CPU caches}
%
\subsection{Placement policies}
Cache placement policies determine where a specific memory block can be loaded into
the cache. The choice of placement policy influences the cache architecture and
its control logic - affecting the overall complexity and performance of the system.
Each policy involves trade-offs between speed, by the means of reducing cache misses and
thrashing, and hardware costs related to the size and design of the hardware.

\vspace{10px}
\noindent Memory and cache configuration for all examples in this section:
\begin{itemize}
	\item Memory size: 1 KiB
	\item Cache size: 32 bytes
	\item Cache block size: 4 bytes
	\item Number of lines: 8
\end{itemize}

\subsubsection{Fully associative cache}
\begin{center}
	\centering
	\includegraphics[width=\textwidth]{figures/02-background/full_ass_mem.pdf}
	\captionof{figure}{Visual representation of fully associative mapping}
	\label{fig:full_ass_mem}
\end{center}

\noindent In the fully associative cache each \textit{cache line} can hold a copy of \textit{any memory location}. The memory address is split into the following bit fields:

\begin{itemize}
	\item \textbf{Offset:} bits used to determine the byte to be accessed from the cache line. In this example there are two bits, that are used to address 4 bytes in the cache line.
	\item \textbf{Tag:} is used by the hardware comparator in the cache hardware design to match the address on cache request. It takes up the rest of the address, in our example 8 bits.
\end{itemize}


\noindent This configuration minimizes the chances of cache misses due to conflicts, potentially improving performance. However, this type of cache requires complex hardware for
searching and managing, as it needs to check all entries simultaneously - this means that there are needed as many comparators as there are bits in the \textit{tag} field.
This means that such implementation is unpractical and seldom used for larger caches \cite{whatevery}, this results in high power consumption and greatly increased die size.

\subsubsection{Set associative cache}
\begin{center}
	\centering
	\includegraphics[width=\textwidth]{figures/02-background/set_ass_mem.pdf}
	\captionof{figure}{Visual representation of 4-way set associative mapping}
	\label{fig:set_ass_mem}
\end{center}

\noindent The set associative cache introduces a concept of a \textit{set} - a collection of more than one cache line sharing the same \textit{tag}.

\begin{itemize}
	\item \textbf{Offset:} bits used to determine the byte to be accessed from the cache line. In this example there are two bits, that are used to address 4 bytes in the cache line.
	\item \textbf{Set:} (sometimes also called \textbf{index}) bits are used to select the cache set comparators. This example presents 4-way set associative cache - meaning that there are 4 lines
		in each set - needing 2 bits to address it.
	\item \textbf{Tag:} is used by the hardware comparator in the cache hardware design to match the address on cache request. It takes up the rest of the address, in our example 6 bits.
\end{itemize}

\noindent This configuration exchanges the flexibility of fully associative placement, for a significantly simplified hardware model. The reduced flexibility comes from the fact that
not every memory address can be cached at the same time, let's imagine that the following addresses are loaded to the cache \texttt{0b0000\_00\_00}, \texttt{0b0001\_00\_00}, \texttt{0b0011\_00\_00}
and \texttt{0b0111\_00\_00}; all of these four lines share the same \textit{set} \texttt{0b00}, loading another address with the same set, for example \texttt{0b1000\_00\_00} would require 
an eviction of one of those lines. Eviction methods will be covered in the (\ref{sec:eviction_policies}) section.

% \noindent This is the most popular type of cache \cite{digitaldesgnandcomp}.

\subsubsection{Directly mapped cache}
In the directly mapped cache each \textit{cache line} can hold a copy of a single \textit{tag}. This policy is equivalent to the set associative cache, with 1-way associativity.
% TODO: image, diagram

%
\subsection{Replacement policies} \label{sec:eviction_policies}
In cache design, the replacement policies describe behavior that is performed when a line needs to get \textit{evicted} from the cache memory.
The choice of replacement policy is a trade-off between cache performance and design complexity required to implement given solution. Low power
devices, such as embedded microcontrollers often opt of simple implementations, such as \textit{random eviction}, or \textit{FIFO}, as these require % TODO: add \ref to these methods
a lower amount of digital logic to select a line to evict, decreasing power consumption. High performance processors, found in PCs, servers, tablets and smartphones,
often use more advanced policies, such as \textit{LRU} and \textit{LFU}.

\subsubsection{Queue based}
% TODO(WHOLE PAR): replace 'policy' - used to often
Queue-based replacement policies use data structures such as queues to manage the replacement process. The most common queue based policy is the
\textit{First In First Out (FIFO)} policy. In this implementation the oldest cache line is replaced first. The main advantages of this policy are  % TODO: replace 'implementation'
its relative ease of implementation and predictable eviction behavior. That simplicity comes at a cost, as this policy does not guarantee optimal cache usage. The queue
based policies do not consider the frequency or recency of the data access when evicting lines, this can lead to a scenario where a data line, that has just been accessed - and due to
temporal locality - is likely to be accessed again, might get prematurely evicted from the cache, simply because it was the first loaded line.

\vspace{10px}\noindent Other queue based replacement policies include: Circular Buffer, Second Chance or Clock algorithms. % TODO: find citations

\subsubsection{Recency based}
Recency based policies prioritize data that has been accessed most recently - leveraging the concept of temporal locality - addressing one of
the main drawback of queue based policies. One of the examples of such policy is \textit{Least Recently Used (LRU)} policy. In this approach, the cache controller
keeps track of the recency of data by maintaining an ordered list of cache lines, where the most recently accessed lines are moved to the front. Due to the need of monitoring
the usage data, LRU (or other frequency based policies) require much more complex hardware designs, increasing power consumption. 

\vspace{10px}\noindent Other recency based replacement policies include: Most Recently Used, LFU with Aging or LRU-K. % TODO: find citations

\subsubsection{Frequency based}
Frequency based policies attempt to leverage the temporal locality of the memory transfers, by prioritizing the data that has been used (read) most frequently.
An example of these policies is the \textit{Least Frequently Used (LFU)}. Similarly to the recency based approaches, the cache controller tracks additional usage counters
for each line, related to the amount of times that the address has been accessed - this increases performance at the cost of increased complexity.

\vspace{10px}\noindent Other frequency based replacement policies include: Adaptive Replacement Cache, Two-Level Adaptive Replacement Cache or Least Frequently Used with Dynamic Aging. % TODO: find citations
%
\subsection{Cache coherency}

The goal of cache coherency is to maintain a consistent state between two or
more separate cache memories. This process can take place both in single-core
systems - in which the state is synchronized between two or more different levels of caches
- and in multicore processors, where the caches are kept in sync between multiple
processors.

\subsubsection{Direct memory access} \label{sec:dma}
% TODO: citations
Direct memory access, also referred to as \textit{DMA}, is a mechanism that
frees up the CPU cycles dedicated to memory operations. It is commonly used 
in one of two configurations:

\begin{itemize}
	\item \textbf{I/O to memory DMA:} In this mode, the CPU initiates the data block % TODO: back this up
		transfer, after which the DMA controller takes over to move or copy the data.
		While the transfer is ongoing, the CPU remains free to perform other operations.
		The processor receives an interrupt from the DMA controller once the transfer is complete.
		This arrangement allows for multitasking during data transactions.
	\item \textbf{Memory to memory DMA:} In this configuration the peripheral can perform memory % TODO: this too
		transfers completely independently of the CPU. This allows for high speed data transfers
		between different memory regions - often spanning multiple devices - without increasing
		the processor workload.
\end{itemize}

\noindent DMA functionality can be found in a variety of modern devices and
peripherals, starting from the relatively simple implementations found in the
UART and USART controllers, passing through more complex setups in network % TODO: cite some examples of DMA here
interfaces, and ending on high-performance computation accelerators (GPU/NPU)
and storage controllers.

\noindent In the cache coherency context, special care needs to be % TODO: cite something
taken when utilizing and implementing DMA mechanisms, the figure
(\ref{fig:dma_cache_issues}) presents the example of memory $\leftrightarrow$
cache consistency error caused by an DMA access.

\begin{center}
	\centering
	\includegraphics[width=\textwidth]{figures/02-background/dma_cache_issues.pdf}
	\captionof{figure}{Coherency error caused by a DMA transfer}
	\label{fig:dma_cache_issues}
\end{center}

\noindent This example starts off with the main memory and first level data cache - referred to as \texttt{l1d\$}, in sync.
A DMA operation updates the main memory address \texttt{0x8000\_0000} to a new value \texttt{0xFF} - The change is performed directly
in the memory, bypassing the CPU cache system. When CPU accesses this memory address, it retrieves the value from its
cache line, which has not been updated, as the DMA transfer was completely transparent for the processor.
In system design, there are two main ways to address such problems:

\begin{itemize}
	\item \textbf{Hardware approach:} \textit{bus snooping} (sometimes also referred to as\textit{bus sniffing}) is an % TODO: cite bus spoofing
		approach where an additional hardware coherency controller - called \textit{snooper} - monitors the outstanding
		transfers on the bus. When external access to a bus is detected, it sends a signal to the cache system, where the
		effected caches are either flushed, or the invalid lines are marked as \textit{dirty}. The primary advantage of this solution is the reduced
		complexity of the operating system and/or drivers, coupled with the CPU executing fewer instructions - at the cost of increased hardware complexity. % TODO: lots of 'complexity' here, refactor so it flows better
	\item \textbf{Software approach:} in designs where implementing hardware coherency solution is not possible - due to space, power or cost constraints -
		the software solutions must be used. In these cases the operating system, or the device drivers, must keep track of ongoing DMA transfers, and manually
		invoke cache invalidating and flushing instructions when necessary. While such solutions reduce the overall hardware design complexity, they add a
		runtime overhead for the processor.
\end{itemize}


\subsubsection{Symmetric multiprocessing}
% TODO: citations

In the context of CPU's, symmetric multiprocessing (or shared-memory multiprocessing) is a system architecture, where two or more \textbf{identical processors} % TODO: add a footnote that SMP implies homogeneous systems (and CITE this as this is a pretty bold statement lol)
are running \textbf{independently of each other}. In SMP systems processors share a common memory pool, allowing for data sharing and communication
between cores. Such configuration requires that multiple cores share a common bus. Some examples of such buses include \textit{Front-side bus}, \textit{Intel Ultra Path Interconnect} % TODO: add citations here (yes, I'm artifically inflating the bilbiography :))
and \textit{HyperTransport} on the x86 personal computers; examples used in embedded systems are \textit{Arm Advanced Microcontroller Bus Architecture} and \textit{SiFive TileLink}. % TODO: citation for AMBA TODO: XXX: is tilelink valid here?

One of the key benefits of SMP designs is their ability to increase performance in a scalable manner, benefiting from parallelization of workloads among multiple cores. % TODO: really please start citing this
Another advantage of such designs is the increased data throughput between multiple cores, which tend to be physically located close to each other on the silicon die.
This proximity allows for much higher bus clock frequencies, thereby increasing the maximum possible data transfer bandwidth on the shared bus. % TODO: lots of 'bus' here

In the context of cache coherency, processors in SMP systems typically share not only the main memory but also higher-level caches - such as level 2 and level 3 caches.
The contents of first level of cache are usually not shared between cores. Despite L1 being private to each core, coherence must be maintained across all cache levels. % TODO: footnote explaining cache sync vs cache cocherence
This is achieved by using various \textit{coherence protocols}, such as \textbf{MESI} (Modified, Exclusive, Shared, Invalid) and \textbf{MOESI} (Modified, Owned, Exclusive, Shared, Invalid).
This work will not cover the specific implementation details of these protocols. % XXX: nice cliffhanger ;) but rly maybe I should at least to try to put a simplified diagram 'ere?



\chapter{State of the art} % XXX: EXTREMELY_NASTY: i might've misunderstood the point of this section, should _academic_ papers be reviewed here? anyways,
% this section will need some elbow grease to "fit in" into the definition of the "state of the art"

\section{Whole system emulators}

\subsection{Renode Framework}

The Renode Framework is Antmicro’s open-source, permissively licensed (MIT), embedded device simulation framework. Being mostly written in C\#, the framework provides an organized,
easily expandable codebase, rich in object-oriented principles. It can be compiled using both mono and .NET build systems for all major operating systems.
The framework supports a wide range of CPU architectures, including, among others, \textit{ARMv7}, \textit{ARMv8}, and \textit{RISC-V}.
Additionally, Renode offers comprehensive whole-system emulation, allowing for the simulation of complex peripherals and interconnections. This capability enables developers to
create and test realistic embedded system environments, including I/O devices, sensors, and communication interfaces. The platform description files and execution scripts
use simple and easily modificable syntax, allowing for rapid iteration, efficient testing, and seamless integration with various development workflows.

One of the outstanding qualities of the Renode framework is its advanced debugging, logging, and execution tracing systems, such as:
\begin{itemize}
	\item \textbf{Execution tracer:} this subsystem allows for monitoring and saving the traces of all major CPU operations, including execution tracing, memory access logging and
		performed I/O operations.
	\item \textbf{Execution metrics:} a module that allows to measure the quantitative data related to the simulation, including number of accesses to peripherals, number of exceptions and
		the number of executed instructions (including counting of specific opcodes).
	\item \textbf{Execution profiler:} a call stack analysis tool intended for debugging and inspecting the guest virtual machine software.
	\item \textbf{Python hooks:} the framework utilizes a built-in Python API to provide an easy entry point to automate testing, extend the simulator's functionalities, and integrate
		seamlessly with other tools and workflows.
\end{itemize}

\noindent Due to its advanced debugging capabilities, the Renode framework was selected as the emulation framework for this work.

\subsection{QEMU}

QEMU, also called \textit{Quick Emulator} \cite{qemuoriginal}, is a whole-system emulation framework, with KVM support. Originally developed by Fabrice Bellard \cite{qemufabrice} and released
under GNU General Public License v2. It is written entirely in the C programming language, and can compiled for all major operating systems.

It supports most popular CPU architectures and is capable of whole system emulation. However, unlike Renode, the platforms in QEMU are defined in \textit{header files}, which means
that changes to the platform configuration are relatively slow and require recompilation of the project. Another similarity to Renode is that QEMU also provides a set of debugging
and analysis tools. However, instead of using Python as a scripting engine, QEMU implements its own fully custom plugin system, which also requires the user to recompile the
application to apply changes. Its main distinguishing factor is built-in support for userspace and KVM emulation, allowing for very efficient emulation if the simulated software is
compiled for the same architecture and/or operating system. In the context of this thesis, QEMU ships with a \textit{Cache Modelling TCG Plugin} \cite{tcgcachemodelling}.

% XXX: i don't like this, this is supposed to be an overview of the state of the art... and I'm showing off 2 simulators that are somewhat "popular",
% and putting the rest into a "catch-all bin". In the ideal world I'd not do that, but to be frank, I don't really know what I should describe here, as I'm not
% that familiar with them...
\subsection{Other notable simulators}

\begin{itemize} % TODO: add more info here
	\item \textbf{Intel Simics:} is a full-system simulator developed by Wind River, a subsidiary of Intel. Simics supports a wide range of CPU architectures and peripheral models.
		It is released under a proprietary, closed-source license.
	\item \textbf{gem5:} event driven full-system emulator, mainly used in academia and advanced research institutions. Developed by universities, companies and the community, maintained by the
		\textit{Project management committee} \cite{gem5governance}. % TODO: gem5 is mainly used in the academia, it is pretty advanced, but hard to use for a "newcomer", I have no idea how to put this nicely
\end{itemize}

\section{Cache simulators}

\subsection{gem5}

The \textit{gem5} project is a highly flexible, modular platform for computer system architecture research and development, that includes advanced cache modeling features \cite{gem5cachesupport}.
It has been previously used in a number of academic papers discussing the topic of CPU caches \cite{gem5cachecite1, gem5cachecite2, gem5cachecite3}.

This simulator supports a wide range of cache architectures and configurations, offering high flexibility, essential for research and academic institutions \cite{gem5}.
It operates with cycle-level simulation accuracy, allowing for modeling and measuring the timing performance characteristics of cache behavior. This potentially provides a higher level of accuracy
compared to behavioral or functional simulators \cite{gem5accuracy}. \textit{gem5} enables the simulation of complex memory hierarchies, including multi-level caches, which are commonly
used in modern computer architectures \cite{gem5multilevel, cachesimsurv}.

Despite its strengths, the high level of detail and flexibility in can make it complex and difficult to set up. The documentation is extensive, but its high level of detail can be overwhelming,
making it challenging to navigate effectively \cite{gem5hell}. The complex simulation model also requires significant computational resources for simulating complex, real world payloads
\cite{gem5, cachesimsurv}. As an open-source and mainly community and academia backed project, \textit{gem5} relies on community updates and maintenance, which can lead to periods where certain
features may be underdeveloped or not fully supported \cite{gem5hell, gem5maintainers}. While the project uses a modular approach for its modules and hardware models, the project complex codebase,
unstable APIs, frequent functionality breaking changes \cite{gem5hell} result in the fact that \textit{gem5} is very hard to integrate with external tooling.

\subsection{QEMU TCG Cache Modeling plugin} \label{sec:qemu_cache}

The TCG Cache Modeling plugin is a plugin for the QEMU emulator, created as a part of Google Summer of Code \cite{qemucachegsoc}. It supports a variety of cache configuration parameters, such as
cache and block size, eviction policies and optional L2 support \cite{tcgcachemodelling}.

A significant advantage of the plugin is its ease of use in the existing QEMU workflows, since it doesn't require any additional software to be introduced into the development processes.
However, such integration makes it nearly impossible to integrate this cache analysis tool with other simulators. There is no functionality to save the trace files, and process them off-line,
instead, the entirety of the cache model simulation happens entirely within the constraints of the TCG plugin system \cite{qemutcgplugindocs}, thus begin reliant on the QEMU for simulating the
payloads. For users looking for modular cache simulation tools that can be adapted to various other environments, this could pose a significant constraint.

\vspace{10px}
\noindent Efforts have also been made to incorporate cache modeling as a core feature of the QEMU simulator \cite{qemucacheattempt}, but these enhancements have not been merged, and the
source code has not been made publicly available.

\subsection{dineroIV}

DineroIV is a cache simulator for memory reference traces, created by the researchers from \textit{University of Wisconsin, Computer Sciences} \cite{dinero}. 
The cache model supports a variety of configuration parameters, such as overall cache size, block size, eviction methods, and others. These parameters are used to define the structure of
single and multi-level caches. DineroIV is purely a cache simulator, with no system emulation backend - it can be only used to emulate the cache from the memory reference traces.
It has been last updated in 1999 and is not available under open source license.

\section{pycachesim}

Pycahesim is a single-core cache hierarchy simulator with Python API and C backend. It has been developed and released by Regionales Rechenzentrum Erlangen, Hochleistungsrechner (RRZE-HPC)
under a AGPL-3.0 license \cite{pycachesim}. The Python API is straightforward and can be easily integrated into external projects. Since the backend is written in a compiled language, it is
relatively fast. The simulator supports complex, multilevel cache configurations, including exclusive and victim caches. % TODO: relative to what? 
However, pycahesim does not support the differentiation of instruction and data caches, which can limit its utility in more complex simulation scenarios. Additionally, the AGPL
license imposes significant restrictions on the commercial use of the project, resulting in little to no industry adoption. Due to this fact, the simulator remains mostly supported
by academic institutions, resulting in very infrequent updates\footnote{As of June 2024, the last update was released over two years ago.}.

\vspace{10px}
\noindent Efforts have been made to integrate QEMU and pycachesim \cite{pycachesimqemu}.

\subsection{Other notable simulators}
% TODO: move items from the list to the sections, explain every one briefly, put some citations, and it shoooould be goooood
\begin{itemize}
	\item \textbf{DynamoRIO, drcachesim:} hard to extract the cache only
	\item \textbf{Cachegrind:} only supports one level of cache
	\item \textbf{libCacheSim:}
	\item \textbf{SMPcache:} closed source, windows + gui only
	\item \textbf{CMPsim:} academically published, source code not available
	\item \textbf{CASPER:} academically published, source code not available
\end{itemize}


\chapter{Implementation}

\section{Implementing the cache model}

The entirety of the cache model has been implemented in the Python programming language. The motivation for this choice includes the portability of the code, ease of customization and
the ability to rapidly iterate and test different configurations.

\subsection{Model architecture}

\begin{center}
	\centering
	\includegraphics[width=\textwidth]{figures/04-implementation/cache_mdl_arch.pdf}
	\captionof{figure}{Visual representation of the cache model architecture}
	\label{fig:cache_mdl_arch}
\end{center}

\subsubsection{\texttt{Cache}} \label{sec:cache_model}

The Cache class encapsulates the cache memory behavioral model. Besides statistic gathering and helper debug methods, its only public interfaces are methods
providing read/write operations, cache flushing, and constructor with configuration parameters. The cache configuration is derived from the following inputs:
\begin{itemize}
	\item Cache width: width of cache memory - $\log_2(\text{cache size})$
	\item Block width: width of cache block - $\log_2(\text{block size})$
	\item Memory width: memory address width
	\item Lines per set: Set associativity: $2^n,\, n \in \{1, 2, 3, ...\}$; -1 for fully associative, 1 for direct mapping
	\item Replacement policy: FIFO, LRU, LFU or None for random
\end{itemize}

\begin{center}
\centering
\begin{minipage}{\linewidth}
\begin{lstlisting}[
    language=Python,
	morekeywords={self},
    label={lst:cache_ctor},
    caption={\texttt{Cache} constructor with configuration parameters}
    ]
def __init__(
	self,
	name: str,
	cache_width: int,
	block_width: int,
	memory_width: int,
	lines_per_set: int,
	replacement_policy: str
):
'''
    +----------------------------+--------------------+--------------+
    |            TAG                |           SET         |     OFFSET    |
    +----------------------------+--------------------+--------------+
    C                               B                       A               0

    TAG    = [C   : B]
    SET    = [B-1 : A]
    OFFSET = [A-1 : 0]
'''
# Width of the memories
self._cache_width = cache_width
self._block_width = block_width    # [A:0] width
self._memory_width = memory_width  # [C:0] width

# Convert __width of the bus__ to __size in bytes__
self._cache_size = 2 ** self._cache_width
self._block_size = 2 ** self._block_width
self._memory_size = 2 ** self._memory_width

self._num_lines = self._cache_size // self._block_size
self._lines = [CacheLine() for i in range(self._num_lines)]

if lines_per_set == -1:
    # special configuration case for fully associative mapping
    lines_per_set = self._num_lines

if not (lines_per_set & (lines_per_set - 1) == 0) or lines_per_set == 0:
    raise Exception('Lines per set must be a power of two (1, 2, 4, 8, ...)')

self._lines_per_set = lines_per_set
self._sets = self._num_lines // lines_per_set
self._set_width = int(math.log(self._sets, 2))  # calculate [B:A] width

self._replacement_policy = replacement_policy

# Statistics
self.misses = 0
self.hits = 0
self.invalidations = 0
self.flushes = 0
\end{lstlisting}
\end{minipage}
\end{center}

\subsubsection*{Read and write cache operations}
\noindent The \texttt{read} and \texttt{write} methods are the primary interfaces for the cache and have been implemented as follows:

\begin{center}
\centering
\begin{minipage}{\linewidth}
\begin{lstlisting}[
    language=Python,
	morekeywords={self},
    label={lst:cache_write_read},
    caption={\texttt{Cache} write and read interfaces}
    ]
def read(self, addr: int) -> None:
    self.printd(f'reading from cache @ {bin(addr)} (set {self._addr_get_set(addr)})')
    line, _, _ = self._line_lookup(addr)

    if line and not line.free:
        self.printd('[read] rhit')
        self.hits += 1
        line.use_count += 1
    else:
        self.printd('[read] rmiss')
        self.misses += 1
        line = self._load(addr)

def write(self, addr: int) -> None:
    self.printd(f'[write] writing to cache @ {hex(addr)}')
    line, _, _ = self._line_lookup(addr)

    if line:
        self.printd('[write] whit')
        self.hits += 1
    else:
        self.printd('[write] wmiss')
        self.misses += 1
        line = self._load(addr)
\end{lstlisting}
\end{minipage}
\end{center}

\noindent The code responsible for finding matching cache line for the given address has been extracted to the internal \footnote{The Python programming language does not have a
built-in mechanism for enforcing access rights to fields in a class. However, PEP 8 specifies that variables starting with '\texttt{\_}' should be used internally in the class
\cite{pep8}.} \texttt{\_line\_lookup} method:

\begin{center}
\centering
\begin{minipage}{\linewidth}
\begin{lstlisting}[
    language=Python,
	morekeywords={self},
    label={lst:line_lookup},
    caption={\texttt{Cache} line lookup}
    ]
def _line_lookup(self, addr: int) -> tuple[CacheLine, list, int] | None:
    tag = self._addr_get_tag(addr)
    set_index = self._addr_get_set(addr)
    lines_in_set = self._get_lines_for_set(set_index)

    line = None
    for line_candidate in lines_in_set:
       if line_candidate.tag == tag:
            line = line_candidate
            break

    return line, lines_in_set, tag if line else None
\end{lstlisting}
\end{minipage}
\end{center}

\noindent If a matching line is found, the function returns a reference to this line, other lines in the same set and the line tag. If no match can be made, the method returns \texttt{None}. This
method uses helper functions to extract the bit-fields from the memory address:


\begin{center}
\centering
\begin{minipage}{\linewidth}
\begin{lstlisting}[
    language=Python,
	morekeywords={self},
    label={lst:bitfield_hepers},
    caption={Bitfields helper functions}
    ]
def create_mask(start_bit: int, end_bit: int) -> int:
    num_bits = end_bit - start_bit + 1
    mask = (1 << num_bits) - 1
    mask <<= start_bit
    return mask

def extract_bits(value: int, start_bit: int, end_bit: int) -> int:
    mask = create_mask(start_bit, end_bit)
    extracted_bits = (value & mask) >> start_bit
    return extracted_bits

class Cache:
    def _addr_get_tag(self, addr: int) -> int:
        ''' Extract [C:B] from the address '''
        start = self._block_width + self._set_width
        end = self._memory_width
        return extract_bits(addr, start, end)
    
    def _addr_get_set(self, addr: int) -> int:
        ''' Extract [B-1:A] from the address '''
        start = self._block_width
        end = self._block_width + self._set_width - 1
        return extract_bits(addr, start, end)
    
    def _addr_get_offset(self, addr: int) -> int:
        ''' Extract [A-1:0] from the address '''
        start = 0
        end = self._block_width - 1
        return extract_bits(addr, start, end)
\end{lstlisting}
\end{minipage}
\end{center}

\subsubsection*{Cache line loading}
\noindent The line loading is facilitated by the \texttt{\_load} method:

\begin{center}
\centering
\begin{minipage}{\linewidth}
\begin{lstlisting}[
    language=Python,
	morekeywords={self},
    label={lst:cache_load},
    caption={\texttt{Cache} load method}
    ]
def _load(self, phys_addr) -> CacheLine:
    self.printd(f'[load] loading @ {hex(phys_addr)} to cache')
    tag = self._phys_addr_get_tag(phys_addr)
    set_index = self._phys_addr_get_set(phys_addr)
    lines_in_set = self._get_lines_for_set(set_index)

    free_line_index = next((index for index, obj in enumerate(lines_in_set) if obj.free), None)
    if free_line_index is not None:
        index = free_line_index
    else:
        index = self._select_evicted_index(lines_in_set)
        self.invalidations += 1
        self.printd(f'[load] set: {set_index}, invalidating line: {index}')

	self.printd(f'[load] loading a new line to set: {set_index} with index: {index}')
    victim = lines_in_set[index]
    victim.free = False
    victim.tag = tag
    return victim
\end{lstlisting}
\end{minipage}
\end{center}

\subsubsection*{Cache flushing}
\noindent Cache flushing is implemented by re-initializing the internal cache state:

\begin{center}
\centering
\begin{minipage}{\linewidth}
\begin{lstlisting}[
    language=Python,
	morekeywords={self},
    label={lst:cache_flush},
    caption={\texttt{Cache} flush method}
    ]
def flush(self) -> None:
    self.printd('[flush] flushing cache')
    self.flushes += 1
    self._lines = [CacheLine() for i in range(self._num_lines)]
\end{lstlisting}
\end{minipage}
\end{center}

\subsubsection*{Debugging features}
\noindent The cache class also comes with a set of debugging helpers:
\begin{itemize}
    \item \texttt{print\_addr\_info}: prints address-related information in the specified format (binary or hexadecimal). It displays the address, tag, set, and offset values.
    \item \texttt{print\_cache\_info}: prints details about the cache configuration, including the width of the tag, set, and block fields, as well as the size and number of various cache components.
    \item \texttt{print\_hmr}: prints the hit-miss ratio along with the number of hits, misses, and invalidations.
    \item \texttt{print\_debug\_lines}: prints debug information for each cache line, including the tag, whether the line is free, and its use count.
\end{itemize}

\subsubsection{\texttt{CacheLine}}

The \texttt{CacheLine} class represents a cache line entry:

\begin{center}
\centering
\begin{minipage}{\linewidth}
\begin{lstlisting}[
    language=Python,
	morekeywords={self},
    label={lst:cacheline},
    caption={\texttt{CacheLine} class}
    ]
class CacheLine:
    def __init__(self):
        self.tag = 0b0
        self.use_count = 0x0
        self.free = True
\end{lstlisting}
\end{minipage}
\end{center}

\noindent
\begin{itemize}
    \item \textbf{Tag:} Binary identifier distinguishing the data block, extracted from the physical address.
    \item \textbf{Use Count:} Counter for replacement policies; tracks accesses (LFU), recency (LRU), or order (FIFO).
    \item \textbf{Free:} Boolean flag; \texttt{True} if the line is free for new data, \texttt{False} if it contains valid data, and will need to be evicted.
\end{itemize}


% TODO: these should be moved somewhere else, as we need renode traces - or I could just forward-reference or sth
\section{Cache interfaces}

The \textit{cache model} (\ref{sec:cache_model}) does not implement any execution, log parsing, and validation logic. This is a deliberate choice to maintain high reusability and modifiability
of the emulation core component. To test and use the cache model, two interfaces were implemented:

\begin{itemize}
	\item \textbf{DummyLogInterface:} a "dummy" interface, intended for usage in test cases. Does not implement any log parsing, instead, the memory accesses are passed as a list of
        memory operations. Does not provide any user configurable cache parameters options.
    \item \textbf{RenodeLogInterface:} Interface intended to be used with Renode's \textit{ExecutionTracer} log format. Implements a command-line interface, allowing for easy cache
        configuration changes, as well as a set of preconfigured cache configurations for a selection of devices.
\end{itemize}


\subsection{\texttt{DummyLogInterface}}

\begin{center}
\centering
\begin{minipage}{\linewidth}
\begin{lstlisting}[
    language=Python,
	morekeywords={self},
    label={lst:dummyinterface},
    caption={\texttt{DummyLogInterface} implementation}
    ]
class DummyLogInterface:
    def __init__(self, cache: Cache):
        self.cache = cache
        self.count_insn_read = 0
        self.count_mem_read = 0
        self.count_mem_write = 0
        self.count_io_read = 0
        self.count_io_write = 0

    def simulate(self, data: dict):
        for access in data:
            for type, addr in access.items():
                match type:
                    case 'mw':
                        self.count_mem_write += 1
                        self.cache.write(addr)
                    case 'mr':
                        self.count_mem_read += 1
                        self.cache.read(addr)
                    case 'ior':
                        self.count_io_read += 1
                    case 'iow':
                        self.cache.flush()
                        self.count_io_write += 1
                    case _:
                        raise ValueError('Unsupported memory operation!')
\end{lstlisting}
\end{minipage}
\end{center}

\noindent Example usage:

\begin{center}
\centering
\begin{minipage}{\linewidth}
\begin{lstlisting}[
    language=Python,
	morekeywords={self},
    label={lst:dummyinterface_usage},
    caption={\texttt{DummyLogInterface} usage}
    ]
cache = Cache(
    name='cache',
    cache_width=6,
    block_width=2,
    memory_width=10,
    lines_per_set=4,
    replacement_policy=None,
)
cinterface = DummyLogInterface(cache)
ops = [
    {'mr': 0b000000_00_00},
    {'mw': 0b000000_00_00},
    {'mr': 0b000000_10_00},
    {'ior': 0b000000_11_00},    
]
cinterface.simulate(ops)
\end{lstlisting}
\end{minipage}
\end{center}

\noindent The main goal of this interface is to wrap the cache model, allowing it to be mainly used later to create and run test scenarios, find and remove implementation errors, and
serve as a test bench for developing new functionalities. It also serves as a generic implementation, that can derived from, and used to aid in connecting this cache model simulator
to other simulators. On its own this model does not, neither it aims to, provide any external trace interface.

\subsection{\texttt{RenodeLogInterface}}

The \texttt{RenodeLogInterface} is an interface allowing for usage of the cache model with the execution trace files generated from the Renode Framework \textit{ExecutionTracer} logging module.
This functionality can be enabled by adding these lines to the platform script file\footnote{Creating, configuring and working with Renode virtual platforms is described in the
(\ref{app:creating_renode_platforms}) section.}:

\begin{center}
\centering
\begin{minipage}{\linewidth}
\begin{lstlisting}[
    label={lst:renode_enabling_exectracer},
    caption={Enabling \texttt{ExeuctionTracer}\protect\footnotemark}
    ]
<cpu_name> MaximumBlockSize 1
<cpu_name> CreateExecutionTracing "tracer" $ORIGIN/trace.log PCAndOpcode
tracer TrackMemoryAccesses
\end{lstlisting}
\end{minipage}
\end{center}
\footnotetext{The \texttt{<cpu\_name>} must be changed to the name of the CPU on which the execution tracer should be activated.}

\noindent Running a Renode simulation with the tracer enabled will generate a \texttt{trace.log} file:
\begin{center}
\centering
\begin{minipage}{\linewidth}
\begin{lstlisting}[
    label={lst:exec_tracer_example},
    caption={Example ExecutionTracer output}
    ]
0x8002110: 0x00B78633
0x8002114: 0x4210
MemoryRead with address 0x8046FC4
0x8002116: 0x973E
0x8002118: 0x0791
0x800211A: 0xC310
MemoryIOWrite with address 0x100B02AC
0x800211C: 0xFED799E3
0x800210E: 0x6018
MemoryRead with address 0x8046CF0
0x8002110: 0x00B78633
0x8002114: 0x4210
MemoryRead with address 0x8046FC8
\end{lstlisting}
\end{minipage}
\end{center}

\noindent The \texttt{RenodeLogInterface} processes these execution trace logs, extracting memory access events for the cache simulation. In the following section, we will review % TODO: improve this
the source code for the \texttt{RenodeLogInterface}, focusing on reading trace files, interfacing with the cache model and streamlining cache parameters configuration.

\subsubsection*{Cache memory configuration}

In order to streamline the cache configuration, this application allows for two ways of ways of cache model configuration:
\begin{itemize}
    \item \textbf{Command line interface parameters:} this method allows users to specify detailed configuration options directly through the command line. Users can define various
        cache parameters such as memory width, cache width, block width, lines per set, and the replacement policy for both L1 instruction and data caches.
    \item \textbf{Presets:} this method loads a predefined set of configuration parameters using a preset name. Presets are useful for standard configurations that are frequently used or
        for quickly setting up the cache model without specifying each parameter individually.
\end{itemize}

\noindent The presets are defined in the interface source code, and are described using the following format:
\begin{center}
\centering
\begin{minipage}{\linewidth}
\begin{lstlisting}[
    caption={Cache configuration presets\protect\footnotemark}
    ]
presets = {
    'fu740.u74': {
        'l1i': Cache('l1i,u74', 15, 6, 64, 4, None),
        'l1d': Cache('l1d,u74', 15, 6, 64, 8, None),
        'flush_opcodes': {
            'i': [0xfc100073],
            'd': [0xfc000073]
        }
    },
    'fe310.e31': {
        'l1i': Cache('l1i,e31', 14, 5, 32, 2, None)
    }
}
\end{lstlisting}
\end{minipage}
\end{center}
\footnotetext{Parameters that are not available on a given platform can be skipped and not included in the preset.} % XXX: drop this?

\noindent Usage using build in presets: \texttt{./renode\_cache\_mdl.py trace.log presets 'fu740.u74'}, equivalent command line parameters:

\begin{verbatim}
./renode_cache_mdl.py trace.log config    \
    --memory_width 64                     \
    --i_invalidation_opcodes '0xfc100073' \
    --d_invalidation_opcodes '0xfc000073' \
    --l1i_cache_width 15                  \
    --l1i_block_width 6                   \
    --l1i_lines_per_set 4                 \
    --l1d_cache_width 15                  \
    --l1d_block_width 6                   \
    --l1d_lines_per_set 8
\end{verbatim}

\subsubsection*{Parsing \texttt{ExecutionTracer} logs}

The traces generated by the \textit{ExecutionTracer} can reach up to several gigabytes in size\footnote{The \textit{ExecutionTracer} also supports saving traces in a custom binary format. However,
this approach was not used because it would add an additional layer of complexity and reduce the readability of the code.}, which is why it is necessary to parse them line by line. 
After each line is parsed, the cache model state is updated accordingly.

\begin{center}
\centering
\begin{minipage}{\linewidth}
\begin{lstlisting}[
    language=Python,
	morekeywords={self},
    label={lst:renodeinterface},
    caption={\texttt{RenodeLogInterface} simulate implementation}
    ]
def simulate(self):
    """ Simulate the cache structure

    Due to _large_ trace files, parse the file line-by-line.

    Renode ExecutionTracer outputs the following data:

    * `PC`: `OPCODE`
    * Memory{Write, Read} with address `ADDR`
    * MemoryIO{Write, Read} with address `ADDR`
    """
    with open(self.fname, 'r') as f:
        for line in tqdm(f, total=lines):
            if ':' in line and self.l1i is not None:
                self.count_insn_read += 1
                pc, opcode = (int(value.strip(), 16) for value in line.split(":"))
                if opcode in self.invalidation_opcodes:
                    match self.invalidation_opcodes[opcode]:
                        case 'i':
                            self.l1i.flush()
                        case 'd':
                            self.l1d.flush()
                self.l1i.read(pc)
            elif line.startswith('Memory') and self.l1d is not None:
                parts = line.split()
                address = int(parts[-1], 16)
                match parts[0].lower().removeprefix('memory'):
                    case 'iowrite':
                        self._handle_io(self, 'write')
                        self.count_io_write += 1
                    case 'ioread':
                        self._handle_io(self, 'read')
                        self.count_io_read += 1
                    case 'write':
                        self.count_mem_write += 1
                        self.l1d.write(address)
                    case 'read':
                        self.count_mem_read += 1
                        self.l1d.read(address)
                    case _:
                        raise ValueError('Unsupported memory operation!')
\end{lstlisting}
\end{minipage}
\end{center}

\section{Integrated test bench}
The integrated test bench is designed to evaluate the correctness of the cache model through three independent test cases, each verifying a different cache placement method:
set associative, fully associative, and direct mapping. In each case, the cache was configured with a 1 KiB size, 4 byte block size, and 10-bit memory addressing\footnote{This
configuration was arbitrarily chosen with the goal of operating on a small number of cache lines, making it easier to work with as a test example.}, with the only difference
being the mapping method. After the configuration, each test performs a set of memory operations that verify the proper behavior when reading from both valid and empty lines, line
eviction behavior, flushing, and correct set/index matching. A snippet of the test for the set associative cache has been included below:

\begin{center}
\centering
\begin{minipage}{\linewidth}
\begin{lstlisting}[
    language=Python,
	morekeywords={self},
    label={lst:cache_write_read},
    caption={Snippet of integrated testbench - set associative unit test}
    ]
cache = Cache(
    name='set_associative',
    cache_width=6,
    block_width=2,
    memory_width=10,
    lines_per_set=4,
    replacement_policy='FIFO',
)
test = DummyLogInterface()
test.configure_caches(cache)

test.simulate([
    {'mr': 0b000000_00_00},  # Set 0
    {'mr': 0b000000_01_00},  # Set 1
    {'mr': 0b000000_10_00},  # Set 2
    {'mr': 0b000000_11_00},  # Set 3
])
assert test.cache.hits == 0
assert test.cache.misses == 4
assert test.cache.invalidations == 0

test.simulate([
    {'mr': 0b000000_00_11},  # Set 0
    {'mr': 0b000000_01_11},  # Set 1
    {'mr': 0b000000_10_11},  # Set 2
    {'mr': 0b000000_11_11},  # Set 3
])
assert test.cache.hits == 4
assert test.cache.misses == 4
assert test.cache.invalidations == 0

...
\end{lstlisting}
\end{minipage}
\end{center}



\chapter{Cache evaluation analysis}

\section{Payloads}

\subsection{Zephyr based matrix-multiplication}

Matrix multiplication is a fundamental operation in many scientific and engineering applications, including high-performance computing (e.g., finite element methods), machine
learning (e.g., linear algebra), and edge computing (e.g., DSP, edge-AI, cryptography), among others. This section discusses the implementation and of various cache multiplication algorithms,
taking into the account cache usage in each of them.

The algorithms have been implemented using the Zephyr RTOS as the execution platform. This particular RTOS has been chosen due to its support for a wide range of hardware boards,
its user-friendly build system (\texttt{CMake} alongside the \texttt{west} helper), its use of the C programming language, and its robust real-time capabilities.
Moreover, Zephyr is a collaborative project with the Linux Foundation and is the fastest-growing real-time operating system \cite{zephyrlotsofcommits}. It has commercial support
from major vendors in the embedded and edge computing sectors, such as Nordic Semiconductors, NXP, STMicroelectronics, Microchip, and many others \cite{aboutzephyr}. Additionally,
it is backed by major companies in the technology sector, including, among others, Google, Meta, Qualcomm and Intel \cite{zephyrmetagoogle, zephyrmembers}.

\subsubsection{Naive approach}
The naive approach to matrix multiplication involves three nested loops iterating over the rows and columns of the matrices. This method, is called naive, as it is simple, but not cache-friendly
due to its poor data locality.

\begin{center}
	\centering
	\includegraphics[width=0.75\textwidth]{figures/05-analysis/mm_naive.pdf}
	\captionof{figure}{Visual representation of the multiplication using the naive approach}
	\label{fig:mm_naive}
\end{center}

The naive approach involves three nested loops, iterating over the rows columns of the matrices:

\begin{center}
\centering
\begin{minipage}{\linewidth}
\begin{lstlisting}[
	style=lstC,
    caption={Naive matrix multiplication implemented in C programming language}
    ]
for (i = 0; i < SIZE; i++) {
	for (j = 0; j < SIZE; j++) {
		for (k = 0; k < SIZE; k++) {
			c[i * SIZE + j] += a[i * SIZE + k] * b[k * SIZE + j];
		}
	}
}
\end{lstlisting}
\end{minipage}
\end{center}

\noindent If matrices are represented as a 1D array in memory, the naive approach to matrix multiplication results in poor cache performance due to suboptimal data locality.
The naive algorithm frequently jumps between distant memory locations\footnote{Assuming $\text{cache size} \ll \text{matrix size}$.}, leading to cache misses.

\subsubsection{Block based approach}
The block-based approach improves cache performance by dividing the matrices into smaller sub-matrices (blocks) that fit into the cache.

\begin{center}
	\centering
	\includegraphics[width=0.75\textwidth]{figures/05-analysis/mm_block.pdf}
	\captionof{figure}{Visual representation of the multiplication using the block based approach}
	\label{fig:mm_block}
\end{center}

\noindent By dividing the large matrix into smaller matrices, data locality is significantly improved, reducing the number of cache misses. This method is particularly effective for
large matrices, where the naive approach would otherwise result in frequent cache evictions. The larger blocks (A11, A12, B11, B12, etc.) are then multiplied and added together.
Selecting the optimal block size is a trade-off. The block size should be small enough to allow the data to fit into the cache but as large as possible to minimize runtime
overhead. Decreasing the block size results in increased runtime overhead because more instructions need to be executed.

\noindent The block-based algorithm has been implemented as follows:
\begin{center}
\centering
\begin{minipage}{\linewidth}
\begin{lstlisting}[
	style=lstC,
    caption={Block based matrix multiplication in C programming language}
    ]
const int B = BLOCK_SIZE;
for (i = 0; i < SIZE; i += B) {
	for (j = 0; j < SIZE; j += B) {
		for (k = 0; k < SIZE; k += B) {
			/* B x B mini matrix multiplications */
			for (i1 = i; i1 < i + B && i1 < SIZE; i1++) {
				for (j1 = j; j1 < j + B && j1 < SIZE; j1++) {
					for (k1 = k; k1 < k + B && k1 < SIZE; k1++) {
						c[i1 * SIZE + j1] += a[i1 * SIZE + k1] * b[k1 * SIZE + j1];
					}
				}
			}
		}
	}
}
\end{lstlisting}
\end{minipage}
\end{center}

\subsection{Linux kernel}

\section{Cache model verification}

\subsection{Integrated test bench}

\subsection{Verification against QEMU TCG Modeling plugin}
The cache model was verified against the QEMU TCG (Tiny Code Generator) modeling plugin. The plugin provides a framework for simulating various cache configurations and comparing
their performance with the actual hardware model.

\subsection{Verification against hardware}
To ensure the accuracy of the cache model, it was compared against hardware measurements. This involved running identical workloads on both the model and the actual hardware, then
comparing hit rate and miss rate. This 

\subsection{Benchmarks}


\section{Results}


\chapter{Conclusions}
% PZIE: In general this is a good chapter. I would, if I were you, put more focus on comparing various versions of software, rather than hardware design - design choices in hw are typically more involved and rely on more factors, including die size, price etc etc

This thesis has explored the efficacy of using trace-based approaches for evaluating CPU cache usage and performance. Several key points should be noted:

\subsubsection*{Effectiveness of trace-based evaluation}

Trace-based evaluation has been proven to be an effective solution for profiling CPU cache performance.
By utilizing trace data, it is possible to gain detailed insights into cache behavior, such as cache hits, misses, and the overall hit-to-miss ratio (HMR). This method allows
for precise analysis of how different cache configurations impact system performance, enabling the identification of bottlenecks and opportunities for optimization.
For example, data from the matrix multiplication benchmarks clearly showed how different block sizes and cache configurations affect the number of cache misses and the
efficiency of cache usage.

\subsubsection*{Impact of cache size and configuration}

The Linux kernel payload benchmark results observed and described in the section (\ref{sec:linux_boot_perf_conc}) confirm that cache size and configuration significantly impact CPU
performance. Factors such as cache size, block size, and associativity play crucial roles in determining system efficiency. Increasing the cache size generally leads to a reduction
in cache misses, which improves overall system efficiency, at the cost of more resource-heavy hardware design. Additionally, optimizing block size and cache associativity can
further enhance performance.
% PZIE: it's not really complicated - it's the same design, but bigger


\subsubsection*{Simulation and real hardware correlation}

This work has confirmed the correlation of the CPU cache behavior between the real hardware platform and the simulated virtual platform and cache model implemented in this work. By
using virtual platforms, developers are provided with an immediate feedback on the impact of code changes on cache performance. This proactive approach to performance optimization
reduces the need for extensive post-deployment tuning, leading to faster and easier development processes.
% PZIE: what is "post deployment tuning"?

\subsubsection*{Practical implications for system design}

The thesis provides a set of findings that have practical implications for system design:

\begin{itemize}
	\item \textbf{Improved cache utilization:} the insights gained from trace-based evaluations can be used to improve cache utilization in various systems. By having a toolset to
		quantitatively and qualitatively measure cache performance, developers can optimize and improve their code and algorithms.
		% PZIE: Qualitatively? How so?
	\item \textbf{Informed hardware design:} the methodology implemented in this work can be adopted for gathering better understanding of the various cache hardware design elements -
		improving the design process
	\item \textbf{Benchmarking and validation:} the established benchmarks and methodologies can be adopted for future hardware validation and implementation of 
	% PZIE: UNFINISHED SENTENCE
\end{itemize}

\section{Future work}

Future research could extend this work by exploring trace-based evaluation in more complex and diverse computing environments. Areas for further investigation include:

\subsubsection*{Multi-core and multi-level cache environments}

This work has not examined the cache behavior in multi-core systems. Further work could be done to extend the cache model with cache coherency protocols, such as MESI or MOESI.
Additionally, the cache model implemented in this work could be further enhanced by adding support for multi-level caches (L2, L3).


\subsubsection*{Advanced cache models}

The model implemented in this work only implements the traditional cache behavior. Further work can be put into implementing advanced cache mechanisms, such as
the Trace Assisted Caching \cite{usingtrace} and dynamic line replacement algorithms \cite{dynamiceviction}.

\subsubsection*{Hardware integration}

Further research can be conducted to explore the possibility of obtaining memory traces from real hardware platforms. This could enable developers and hardware designers to use the
cache model implemented in this work to verify a broader range of devices, enhancing its applicability and accuracy. Additionally, integrating real-world data could lead to more
robust and reliable cache optimization techniques.


%--------------------------------------
% Literatura
%--------------------------------------

\bibliographystyle{unsrt}{\raggedright\sloppy\small\bibliography{bibliografia}}

%--------------------------------------
% Dodatki
%--------------------------------------

\cleardoublepage\appendix%
\newpage
% Removing all includes from this section breaks some conditional
\if

\lstlistoflistings
\clearpage
\listoffigures

\begin{appendices}
   \chapter{Definitions}
% PZIE: Npt sure what you want to achieve here? Do you need this? Remember you can link to the docs as well
\section{Creating and configuring Renode virtual platform configuration files} \label{app:creating_renode_platforms}

The Renode framework uses two types of files to define and configure the simulated virtual platform:

\begin{itemize}
	\item \textbf{Platform description}:
	\item \textbf{Renode scripts}:
\end{itemize}

\end{appendices}

%--------------------------------------
% Informacja o prawach autorskich
%--------------------------------------

\ppcolophon

\end{document}
