%%%%%%%%%%%%%%%%%%%%%%%%%%%%%%%%%%%%%%%%%%%%%%%%%%
%% Bachelor's & Master's Thesis Template        %%
%% Copyleft by Dawid Weiss & Marta Szachniuk    %%
%% Faculty of Computing and Telecommunication   %%
%% Poznan University of Technology, 2020        %%
%%%%%%%%%%%%%%%%%%%%%%%%%%%%%%%%%%%%%%%%%%%%%%%%%%


% Szkielet dla pracy licencjackiej pisanej w języku polskim.

\documentclass[english,masters,a4paper,oneside]{ppfcmthesis}


\usepackage[utf8]{inputenc}
\usepackage[OT4]{fontenc}

% Fix footnotes to the bottom of the page
\usepackage[bottom]{footmisc}

% Better table formatting
\usepackage{makecell}               % Makecell for breaking a long text
\renewcommand{\cellalign}{tl}       % Makecell left alignment
\renewcommand{\arraystretch}{1.75}  % More vertical padding

% Code snippets
\usepackage{listings}
\usepackage{color}

\usepackage{multicol}

\usepackage{caption}
\usepackage{subcaption}
\usepackage{multirow}

\usepackage{pdfpages}

\definecolor{codegreen}{rgb}{0.00,0.60,0.00}
\definecolor{preprocesorbrown}{rgb}{0.50,0.25,0.23} 
\definecolor{codelightblue}{rgb}{0.30,0.70,0.60}

\lstset{frame=tb,
  aboveskip=3mm,
  belowskip=3mm,
  showstringspaces=false,
  columns=flexible,
  basicstyle={\small\ttfamily},
  numbers=none,
  numberstyle=\tiny\color{gray},
  keywordstyle=\color{codelightblue},
  commentstyle=\color{codegreen},
  stringstyle=\color{preprocesorbrown},
  breaklines=true,
  breakatwhitespace=true,
  tabsize=2,
  captionpos=b
}

% Lists without bulletpoints
\usepackage{enumitem}

\lstdefinestyle{lstC}
{
	language = C,
	keywordstyle     =   \color{blue}\ttfamily,
  	commentstyle     =   \color{codegreen}\ttfamily,
  	stringstyle      =   \color{red},
  	emphstyle        =   \color{codelightblue}\ttfamily,
  	directivestyle   =   \color{preprocesorbrown}\ttfamily,
     morekeywords   =   {inline, target_ulong, uint32_t, uint64_t, no_inline, BOOL, _Bool},
}

\lstdefinestyle{lstCsharp} 
{
	language = C++, % No C# support
	keywordstyle     =\color{blue}\ttfamily,
  	commentstyle     =\color{codegreen}\ttfamily,
  	stringstyle      =\color{red},
  	emphstyle        =\color{codelightblue}\ttfamily,
  	directivestyle   =\color{preprocesorbrown}\ttfamily,
  	morekeywords     ={var,async,await,using,byte,string,get,set,uint, checked},
  	emph             ={Descendants,First,FormUrlEncodedContent,HttpClient,
  					   HttpWebRequest,HttpWebResponse,JObject,JProperty,
  					   KeyValuePair,Length,List,OfType,
  					   PropertyChangedEventHandler,Stream,StreamReader,
  					   Value,WebRequest,Where}
}

%--------------------------------------
% Strona tytułowa
%--------------------------------------

% Autorzy pracy, jeśli jest ich więcej niż jeden
% wstaw między nimi separator \and
\author
{%
   Patryk Kościk \album{144635}
}
\authortitle{}                                % Do not change.

\title
{%
   Analysis of Trace-Based Evaluation of Cache Usage on the Example of the Renode Framework
}

% Your supervisor comes here.
\ppsupervisor{dr inż. Mariusz Naumowicz} 

% Year of final submission (not graduation!)
\ppyear{2024}                                 


\begin{document}

% Front matter starts here
\frontmatter\pagestyle{empty}%
\maketitle\cleardoublepage%

%--------------------------------------
% Miejsce na kartę pracy dyplomowej
%--------------------------------------

\thispagestyle{empty}\vspace*{\fill}%
\begin{center}Tutaj będzie karta pracy dyplomowej;\\oryginał wstawiamy do wersji dla archiwum PP, w pozostałych kopiach wstawiamy ksero.\end{center}%
\vfill\cleardoublepage%


% \thispagestyle{empty}\vspace*{\fill}%

\begin{vplace}

\begin{center}
   \huge{\textit{Abstract}}
\end{center}

% TODO: proofread
In the context of modern computing, CPU cache plays a pivotal role in defining
system performance across both complex computing systems and edge/embedded solutions.
%
Consequently, considerable effort is being invested in enhancing cache
implementations and various optimizations, aimed at supporting both
extensive workloads, such as large machine learning models, and smaller, more
time-sensitive workloads, such as improving latencies in real-time operating
systems.
%
This work will focus on determining whether using trace-based approaches is an
effective solution for profiling CPU cache usage and performance.
%
Additionally, this study will examine the effects of cache size and configuration
on processing bottlenecks, offering insights into how these factors influence
overall system performance.

\end{vplace}

\newpage

\begin{vplace}

\begin{center}
   \huge{\textit{Streszczenie}}
\end{center}

% TODO: proofread
W kontekście nowoczesnych systemów komputerowych, pamięć podręczna procesora
odgrywa kluczową rolę w wydajności systemu, zarówno w złożonych
systemach obliczeniowych, jak i rozwiązaniach brzegowych/wbudowanych.
%
W związku z tym pokładane jest dużo pracy w ulepszanie implementacji pamięci
podręcznej i różne optymalizacje, mające na celu wspieranie zarówno złożonych
obliczeniowo zadań, takich jak duże modele uczenia maszynowego, jak i
mniejszych, bardziej wrażliwych na czas obciążeń, takich jak optymalizacja
opóźnień w systemach operacyjnych czasu rzeczywistego.
%
Ta praca skupi się na określeniu, czy wykorzystanie podejścia opartego
na śledzeniu jest skutecznym rozwiązaniem do profilowania wykorzystania i
wydajności pamięci podręcznej procesora.
%
Dodatkowo, to badanie zbada wpływ rozmiaru i konfiguracji pamięci podręcznej na
wąskie gardła przetwarzania, oferując wgląd w to, jak czynniki te wpływają na
ogólną wydajność systemu.

\end{vplace}

\newpage

%--------------------------------------
% Spis treści
%--------------------------------------

\pagenumbering{Roman}\pagestyle{ppfcmthesis}%
\tableofcontents* 
\cleardoublepage % Zaczynamy od nieparzystej strony

%--------------------------------------
% Rozdziały
%--------------------------------------

%Najwygodniej jeśli każdy rozdział znajduje się w oddzielnym pliku
\mainmatter%

\chapter{Introduction}

\section{Motivation and goals of the thesis}
%

\section{Thesis organization}
%


%--------------------------------------
% Literatura
%--------------------------------------

\bibliographystyle{unsrt}{\raggedright\sloppy\small\bibliography{bibliografia}}

%--------------------------------------
% Dodatki
%--------------------------------------

\cleardoublepage\appendix%
\newpage
% Removing all includes from this section breaks some conditional
\if

% \begin{appendices}
%    % \chapter{Definitions}

\section{Creating and configuring Renode virtual platform configuration files} \label{app:creating_renode_platforms}

The Renode framework uses two types of files to define and configure the simulated virtual platform:

\begin{itemize}
	\item \textbf{Platform description}:
	\item \textbf{Renode scripts}:
\end{itemize}

% \end{appendices}

%--------------------------------------
% Informacja o prawach autorskich
%--------------------------------------

\ppcolophon

\end{document}
